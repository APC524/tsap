%! TEX root = design_doc.tex


\subsection{Model fitting}

For a given time series, there are several potential classes of models that could be used to describe its variation. Among them, the most frequent used are the autoregressive(AR) model, integrated(I) model and the moving-average(MA) model. The combination of these classes lead to the autoregressive moving-average(ARMA) model, the autoregressive integrated moving-average(ARIMA) model and the autoregressive fractionally integrated moving-average(ARFIMA) model. The user could choose the model class he/she wish to fit based on his understanding of the input time series.

\subsection{Optimization}
After the model type being fixed, there would be certain loss function regarding the selected model. The key of model fitting is to decide the parameters, which are derived by minimizing the loss function. Therefore, certain optimization techniques should be applied during the fitting. In our software, the user could choose the optimization tool between gradient descent, Newton method and stochastic gradient descent according to the scale of the problem.

\subsection{Inference}
After model fitting, in order to provide the user with more overall information about the model, our software also do inference work regarding the parameters. The software would construct confidence interval of given level for each parameter, conduct test to decide the non-zero parameters and calculate the corresponding P-value. This could help to identify the pattern of the model and help the user develop deeper understanding of the given time series. 

\subsection{Clustering}

When a large number of time series are given, some of them may share similar patterns since they might be commonly affected by several intrinsic factors. With clustering techniques, the software would be able to identify the similar time series, i.e. stocks with similar variation, and divide them into groups. This would help the user gain a better knowledge of the stocks and their patterns.


\subsection{Visualization}
The software is able to predict the future price for each stock based on the fitted model. Hence, in the data visualization part, it would plot the estimated future price and an prediction interval along with the previous price that is already known. Together, some trading strategy would be made based on the prediction. The software could provide the user with the selling or buying point of certain stocks.
The model related functions are written in \texttt{model.py}, which contains two model classes: AR and MA. They are used to fit an AR or MA model by given parameters, to compute the log-likelihood and gradients given an AR or MA model and the data and to give predictions. The functions are introduced below.
Firstly, you can fit an AR or MA model by given parameters. The inputs are as follows: 
\begin{enumerate}
\item \texttt{lag}: the lag parameter for the fitted model, a number;
\item \texttt{phi}: the coefficients for the fitted model, a \texttt{numpy} array whose dimension is the \texttt{lag} given above;
\item \texttt{sigma}: the variance of noise for the fitted model, a number;
\item \texttt{intercept}: the intercept of noise for the fitted model, a number.
\end{enumerate}
The output is a model object.
To fit the model, we can run
\begin{lstlisting}[language=Python]
AR_model = AR(lag=3, phi=np.array([[1],[1],[1]]), sigma=1, intercept=0.1)
MA_model = MA(lag=3, phi=np.array([[1],[1],[1]]), sigma=1, intercept=0.1)
\end{lstlisting}
Secondly, you can compute the log-likelihood and gradients given an AR or MA model and the data. The input is as follows:
\begin{enumerate}
\item \texttt{data}: One or several time series. The input data should be a 2-dimensional array. Each row of data represents a time series over the time, e.g. the stock return over the year of a single stock. The number of rows should be the number of time series.
\end{enumerate}
The output is a number for the computed loglikelihood and a tuple for the computed gradients. This function is public but mostly it serve for the \texttt{Solver} class. 
To compute the log-likelihood and the gradients, we can run 
\begin{lstlisting}[language=Python]
AR_llh, AR_grads= AR_model.loss(data)
MA_llh, MA_grads= MA_model.loss(data)
\end{lstlisting}
Finally, you can make prediction based on a fitted model. The input is as follows:
\begin{enumerate}
\item \texttt{data}: the time series before the first prediction. The number of columns (the length of the time series) should exceed the given lag for a reasonable prediction;
\item \texttt{nstep}: the number of predictions.
\end{enumerate}
To make predictions, we can run
\begin{lstlisting}[language=Python]
AR_future=AR_model.predict(data,nstep=5)
MA_future=MA_model.predict(data,nstep=5)
\end{lstlisting}




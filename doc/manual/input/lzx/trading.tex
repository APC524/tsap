%!TEX root = ../../manual.tex

The functions for trading is written in \texttt{trading.py}, which consists of four functions, i.e., \texttt{signal\_generation}, \texttt{profit\_loss}, \texttt{trade}, and \texttt{rolltrade}.  In the following, we give a detailed introduction of the usage of these functions.

Firstly, \texttt{signal\_generation} returns the trading signal, like what time to buy and what time to sell. The input includes
\begin{enumerate}
\item \texttt{X}: a one dimensional \texttt{numpy} array, which is the price series.
\item \texttt{window}: a positive integer that specifies the largest length of one trade. If \texttt{window}=5, that means there's at most one trade activity within 5 days. The default number is 2.
\end{enumerate}
The output is a one dimensional numpy array, whose elements are 1 (buying signal), -1 (selling signal), 0 (no signal).
To use these function, we can run 
\begin{lstlisting}[language=Python]
import trading 
signal = trading.signal_generation(X, window = 5) 
\end{lstlisting}

Secondly, \texttt{signal\_generation} returns the profit based on given trading signals. The input includes
\begin{enumerate}
	\item \texttt{X}: a one dimensional \texttt{numpy} array, which is the price series.
	\item \texttt{signal}: a one dimensional \texttt{numpy} array that contains trading signals. Usually, it takes the output of \texttt{signal\_generation}.
	\item \texttt{money}: a positive float number, which is the initial wealth. The default number is 1.
\end{enumerate}
The output is a one dimensional numpy array, which is the time series of profit at different time points.
To use these function, we can run 
\begin{lstlisting}[language=Python]
import trading 
signal = trading.signal_generation(X, window = 5) 
profit = trading.profit_loss(Y, signal, money = 100) 
\end{lstlisting}

Next, \texttt{trade} returns the profit in certain time period. Specifically, given a fitted model, it uses this model to forecast future returns based on historical returns, which are obtained by transfering historical prices 
to historical returns. After that, it again transfers the returns to prices and then generates trading signal 
based on predicted prices. At last it calculates the profit based on the real price series and the trading signals. The input includes
\begin{enumerate}
	\item \texttt{X}: a one dimensional \texttt{numpy} array, which is historical price.
	\item \texttt{Y}: a one dimensional \texttt{numpy} array, which is future prices.
	\item \texttt{M}: a class from \texttt{model.py}, which is a fitted model.
	\item \texttt{nstep}: an positive integer, which specifies how many steps to predict.
    \item \texttt{window}: an positive integer that specifies the largest length of one trade. If \texttt{window}=5, that means there's at most one trade activity within 5 days. 
	\item \texttt{money}: a positive float number, which is the initial wealth. The default number is 1.0.
\end{enumerate}
The output is a one dimensional numpy array, which is the time series of profit at different time points.
To use these function, we can run 
\begin{lstlisting}[language=Python]
import trading
import model
AR_model = model.AR(lag=3, phi=np.array([[1],[1],[1]]), sigma=1, intercept=0.1)
profit = trading.trade(X, Y, M = AR_model, nstep = 10, window = 5, money = 100) 
\end{lstlisting}


Lastly, \texttt{rolltrade} returns the profit, trading signal, and the predicted prices by rolling trading in the whole time period, which is an extension of \texttt{trade}. It will repeat the following procedure until the data is used up. Specifically, given a fitted model, it uses this model to forecast future returns based on historical returns, which are obtained by transfering historical prices 
to historical returns. After that, it again transfers the returns to prices and then generates trading signal 
based on predicted prices. At last it calculates the profit based on the real price series and the trading signals. The input includes
\begin{enumerate}
	\item \texttt{X}: a one dimensional \texttt{numpy} array, which is historical price.
	\item \texttt{M}: a class from \texttt{model.py}, which is a fitted model.
	\item \texttt{l}: an integer number, which specifies how many data points are used to train for each trading.
	\item \texttt{nstep}: an positive integer, which specifies how many steps to predict.
	\item \texttt{window}: an positive integer that specifies the largest length of one trade. If \texttt{window}=5, that means there's at most one trade activity within 5 days. 
	\item \texttt{money}: a positive float number, which is the initial wealth. The default number is 1.0.
\end{enumerate}
The output is three one-dimensional numpy array, which are the time series of profit, trading signal, and the predicted prices at different time points.
To use these function, we can run 
\begin{lstlisting}[language=Python]
# import library
import trading 
import model
import numpy as np
import data_processor as dp

# read data and do the preprocessing
data = np.loadtxt("../data/GOOG.csv", delimiter=',')
X = np.array([data[0:100]])
Y = dp.get_return(X)

# initialize the model
lag = 5
sigma = 1.0
intercept = 0.1
phi = np.array([[ 0.04560256],[ 0.0535601 ],[-0.78190871],[ 1.30062633],[ 0.46616754]])
AR_model = AR(lag=lag, phi=phi, sigma=sigma, intercept=intercept)

# solve the model
_, grads = AR_model.loss(Y)
solver = Solver(AR_model, Y, update_rule='sgd_momentum', optim_config={'learning_rate': 1e-6,}, num_epochs=10000, batch_size=1,print_every=10)
solver.train()

# get the trading profit, signal and predicted price
profit, signal, out_pred_price = trading.rolltrade(X, M = AR_model, l = 100, nstep = 20, window = 5, money = 100) 
\end{lstlisting}
%!TEX root = ../../manual.tex

When multiple time series are available, we can analysis the data with modell reduciton methods. The model reduction function is provided in \texttt{reduction.py}. In this file, we provide a class \texttt{Reduction}.

\begin{lstlisting}[language=Python]
class Reduction(object):
    """Callable modal reduction object.
    Example usage:
    xreduction = Reduction(X), X shape [n_features, n_samples], make sure X is zero-mean
    xmean, ux, at, energy_content = xreduction.PCA(n_components=3)
    """
\end{lstlisting}

The member functions of class \texttt{Reduction} consists of Principal Component Analysis (PCA) \cite{jolliffe2002principal}, Independent Component Analysis (ICA) \cite{hyvarinen2000independent}, and Dynamic Mode Decomposition (DMD) \cite{rowley2009spectral} \cite{schmid2010dynamic}. The definitions of three functions are declared below:

\begin{lstlisting}[language=Python]
    def PCA(self, n_components=None):
        """
        Principal component analysis (PCA) of data in matrix
        Inputs:
        n_components: integer, number of principal components
        Returns:
        ux: principal components
        at: principal components coefficients
        energy_content: energy content percentage in the principal components
        """
        
    def ICA(self, n_components, gfunc='logcosh', tol=1e-4, max_iter=200):
        """
        Independent component analysis(ICA) of data in matrix X
        Inputs:
        n_components: integer, number of independent components
        gfunc: string, 'logcosh' or 'exp', default 'logcosh', Non-gaussian function
        tol: float, tolerance of iteration, default 1e-4
        max_iter: integer, maximum iteration steps, default 200
        Returns:
        Ex: array, mean of data
        T: array [n_features, n_features], whitening matrix, st, xtilde = Tx
        A: array [n_features, n_components], mixing matrix, st, xtilde = As
        W: array [n_components, n_features], orthogonal rows, unmixing matrix, st, W = inv(A), s = W*xtilde
        S: array, [n_components, n_samples], source data, st, S = W*Xtilde
        """
        
    def DMD(self, n_components=None):
        """
        Dynamic mode decomposition(DMD) of time series data x(k), find square 
        matrix A such that x(k+1) = Ax(k). Find eigendecomposition of A, and 
        corresponding DMD modes, and DMD eigenvalues.
        """
\end{lstlisting}

To get a better sense of how to analyse real data with this class. Now we will give an example using \texttt{S\&P 500}.
\begin{lstlisting}[language=Python]
import numpy as np
import reduction

# preprocessing, subtract each stock price mean and normalize by std
x = np.copy(SP500)
for i in range(nStock):
    x[i,:] = (x[i,:] - np.mean(x[i,:]))/np.std(x[i,:])
    
# read SP500 data
SP500 = np.genfromtxt('../data/SP500array.csv', delimiter=',')
SP500 = SP500.T

# call and test PCA
n_components = 5
SPreduction = reduction.Reduction(x)
ux, at, energy_content = SPreduction.PCA(n_components)

# call and test ICA
n_components = 10
SPreduction = reduction.Reduction(x)
Ex, T, A, W, S = SPreduction.ICA(n_components)

# call and test DMD
n_components = 20
SPreduction = reduction.Reduction(x)
evals, modes, energy_content = SPreduction.DMD(n_components)
\end{lstlisting}
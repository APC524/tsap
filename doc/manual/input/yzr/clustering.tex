%!TEX root = ../../manual.tex
We also provide realizations of four clustering methods: K-means, spectral clustering, hierarchical clustering, and Gaussian mixture modeling. These methods are organized by  the \texttt{Cluster} class, which is initialized by an input data matrix of dimension $(\texttt{nsample}\times \texttt{nfeature})$ where \texttt{nsample} is the number of data points and \texttt{nfeature} is the number of features.  An object in the class consists of three attributes:\texttt{\_X}, which stores the data matrix,  \texttt{\_nsample}, and  \texttt{\_nfeature}. All clustering functions in this class is  has the form 
\begin{lstlisting}[language=Python]
centroid, labels, clusters =Clustering_method(self, nClusters, maxIter)
\end{lstlisting}
Here \texttt{Clustering\_method} is one of \texttt{Kmeans}, \texttt{H\_clustering},  \texttt{Spectral}, and \texttt{Gaussian\_mixture}. In addition, \texttt{nClusters} is the number of clusters and \texttt{maxIter} is the maximum number of iterations, which has default value $300$. The output consists of three items.  \texttt{centroid}, a $\texttt{nClusters} \times \texttt{\_nfeatures} $ matrix storing the centers of clusters; \texttt{labels} is the predicted labels; \texttt{clusters} stores the index of data points in each cluster. The following example shows how to use our package to cluster the \texttt{S\&P 500} dataset. First we import and standardize the data.
\begin{lstlisting}[language=Python]
import numpy as np
# read SP500 data
SP500 = np.genfromtxt('../data/SP500array.csv', delimiter=',')
SP500 = SP500.T
nStock = len(SP500[:,0])
nTime = len(SP500[0,:])

# preprocessing, standardize data
X = np.copy(SP500)
for i in range(nStock):
    X[i,:] = (X[i,:] - np.mean(X[i,:]))/np.std(X[i,:])
\end{lstlisting}
  The we import our package and fit an K-means clustering model on the data with $3$ clusters.
 \begin{lstlisting}[language=Python] 
from src.cluster import Cluster
model = Cluster(X)
# run K-means

import time

start = time.time()
centroid, labels, clusters = model.kMeans(nClusters = 5)
end = time.time()

print("K-means takes "+str(end-start)+" seconds")
\end{lstlisting}
The output is 
 \begin{lstlisting}[language=Python] 
 K-means takes 0.873986959457 seconds
 \end{lstlisting}
%!TEX root = ../../manual.tex

The class \texttt{OptionPricing} calculates the call option price of an underlying
stock based on the Black-Scholes model.

The following parameters are needed to construct an instance of the
\texttt{OptionPricing} class:
\begin{enumerate}
  \item \texttt{sigma}:
    the volatility of the underlying stock price, which is the standard
    deviation of the stock's returns.
  \item \texttt{K}:
    the strike price of the option.
  \item \texttt{T}:
    the expiry time of the option.
  \item \texttt{r}:
    the risk-free interest rate.
  \item \texttt{Smax}:
    the maximum stock price we want to consider.
\end{enumerate}

The method \texttt{solve\_black\_scholes} calculates the option pricing of an
instance, given the grid size of price and time (\texttt{nS} and \texttt{nt}) as
input parameters. The option price as an function of stock price and time will
be stored in the instance.

After running \texttt{solve\_black\_scholes}, we can use 
\texttt{get\_option\_price} method to get the option price of given underlying
stock price \texttt{S} and time \texttt{t}.

Following is a usage example of the \texttt{OptionPricing} class:
\begin{lstlisting}[language=Python]
goog = np.genfromtxt("../data/GOOG.csv", delimiter=",")
sigma = np.std((goog[1:] - goog[:-1]) / goog[:-1])

option_price = OptionPricing(sigma=sigma, T=90, K=800, r=0.005, Smax=1200)
option_price.solve_black_scholes(nS=100, nt=300)
print(option_price.get_option_price(S=810, t=30))
\end{lstlisting}
